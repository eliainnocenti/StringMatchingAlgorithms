%! Author = eliainnocenti
%! Date = 04/10/23

% Preamble
\documentclass[11pt]{article}

% Packages
\usepackage{amsmath}

%
\title{String Matching Algorithms}
\author{Elia Innocenti}
\date{Novembre 2023}

% Document
\begin{document}
\maketitle

    % Svolgere ed analizzare opportuni esperimenti
    % Scrivere una relazione (in LATEX) che descriva quanto fatto

    % deve contenere:
        % breve introduzione che descrive il problema
        % una breve descrizione delle caratteristiche teoriche degli algoritmi e delle strutture dati utilizzate
        % una valutazione a priori delle prestazioni attese degli algoritmi analizzati sperimentalmente
        % una descrizione degli esperimenti che verranno fatti (non un semplice elenco)
        % la documentazione del codice implementato
        % i risultati sperimentali, sia in tabelle che con grafici l’analisi completa di tali risultati, effettuata in modo critico

    % la teoria:
        % Fa riferimento a quanto studiato nel corso di Algoritmi e Strutture Dati
        % Deve essere solo la parte finalizzata all’esperimento
        % Non va bene un semplice copia/incolla dagli appunti (libro)
        % Bisogna descrivere gli aspetti più importanti e come questi indichino indirettamente quali test eseguire
        % Se serve un teorema, basta mostrarne l’applicazione non serve la dimostrazione

    % documentazione del codice:
        % uno schema del contenuto e delle interazioni fra i moduli uno schema delle classi
        % un’analisi delle scelte implementative effettuate
            % se erano possibili alternative, indicare perché è stata fatta una certa scelta
        % una descrizione dei metodi implementati, indicando in particolare l’input/output e la funzione svolta

    % descrizione degli esperimenti condotti:
        % Bisogna descrivere: i dati utilizzati
            % Se sono stati generato automaticamente, come questo avviene
            % Altrimenti da dove provengono e quali sono le loro caratteristiche
        % Specifiche della piattaforma di test (hardware, sistema operativo);
        % Quali misurazioni vengono effettuate
            % Che tipo di misure
            % Quante volte si eseguono i vari test
        % Come si effettuano le misurazioni (porzioni di codice osservate, numero di run effettuati)

    % presentazione risultati sperimentali:
        % Presentati sia in tabelle che con grafici
            % Le tabelle devono contenere tutti i dati (al limite in un file allegato)
            % I valori nelle tabelle devono avere un numero di cifre significative appropriato (python può fornire numeri con precisione arbitraria)
            % Un grafico serve per evidenziare l’andamento di un valore, ma non sostituisce la tabella
            % A volte possono essere presentati vari grafici per una tabella per mostrare aspetti diversi
            % Un grafico non chiaro o che non mostri qualcosa di interessante è inutile
            % Non importa la bellezza di un grafico
            % Tutti grafici le tabelle e le figure dvono essere
            % Descritti da una didascalia (lunga qb...)
            % Citati nel testo
            % \label{} ... \ref{}

    % analisi dei risultati sperimentali:
        % Un esperimento non è una semplice collezione di dati
        % I risultati di ogni esperimento vanno commentati ed analizzati in modo critico, citando i grafici e le tabelle corrispondenti
        % Nell’analisi si verifica se le ipotesi teoriche vengono verificate con i dati sperimentali
        % Al termine dell’analisi degli esperimenti un paragrafo di conclusioni è spesso utile per sintetizzare i risultati ottenuti

\end{document}